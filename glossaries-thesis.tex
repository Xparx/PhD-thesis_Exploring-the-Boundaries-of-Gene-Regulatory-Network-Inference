% This file adds functionality for glossaries related to Tjarnberg PhD thesis 
%
% documentation: file:///usr/share/doc/texlive-doc/latex/glossaries/glossaries-user.html
% or if running emacs: C-c ? glossaries-user RET
%
%:-------------------------- Glossary/Abbrev./Symbols -----------------------
\usepackage[toc,acronym,style=long]{glossaries} % https://en.wikibooks.org/wiki/LaTeX/Glossary
% \renewcommand{\nomlabel}[1]{\textbf{#1}} % make abbreviations bold
\makeglossaries
\newcommand{\g}{\footnote{For all abbreviations see the glossary on page \pageref{nom}.}} % type "\g" to refer to glossary

% Dual entries with reference to a glosssary entry from an acronym
\usepackage{xparse} % for dual entry simplification
\DeclareDocumentCommand{\newdualentry}{ O{} O{} m m m m } {
  \newglossaryentry{gls-#3}{name={#5},text={#5\glsadd{#3}},
    description={#6},#1
  }
  \newacronym[see={[Glossary:]{gls-#3}},#2]{#3}{#4}{#5\glsadd{gls-#3}}
}

%% Template for dual entry
% \newdualentry{OWD} % label
% {OWD}              % abbreviation
% {One-Way Delay}    % long form
% {The time a packet uses through a network from one host to another} % description

% Acronyms
\newacronym{grn}{GRN}{Gene Regulatory Network}
\newacronym{snr}{SNR}{Signal to Noise Ratio}
\newacronym{bic}{BIC}{Bayesian information criterion}
\newacronym{aic}{AIC}{Akaike information criterion}
\newacronym{rss}{RSS}{residual sum of squares}
\newacronym{cls}{CLS}{constrained least squares}
\newacronym{ols}{OLS}{ordinary least squares}
\newacronym{lsco}{LSCO}{least squares Cut-Off}


% Glossaries
\newglossaryentry{glmnet}
{
  name=Glmnet,
  description={A fast implementation of the algorithms \gls{lasso}, \emph{ridge regression} and the combination of the previous called \emph{elastic net}},
}

% Dual entries
\newdualentry{lasso}
{LASSO}
{Least Absolute Shrinkage and Selection Operator}
{A regression method that penalizes the absolute values of the model parameters.\cite{Tibshirani1996}}

\newdualentry{rni}
{RNI}
{Robust Network Inference}
{A conservative network inference approach that tests all possible networks and selects parameters based on if the parameter is included in all networks that can explain the data with a given confidence\cite{Nordling2013phdthesis}.}

% Symbols
\newglossaryentry{chi2}
{
  name={\ensuremath{\chi^2}},
  description={the chi-square distribution},
  sort=chi2
}
\newglossaryentry{k}
{
  name={\ensuremath{\kappa}},
  description={the condition number},
  sort=k
}
